\documentclass[11pt]{article}

% ==== PACKAGES ==== %
\usepackage{fullpage}
\usepackage{amsmath,amssymb}
\usepackage{epic}
\usepackage{eepic}
\usepackage{hyperref}
\usepackage{listings}
\usepackage{float}

% ==== MARGINS ==== %
\pagestyle{empty}
\setlength{\oddsidemargin}{0in}
\setlength{\topmargin}{-0.8in}
\setlength{\textwidth}{6.8in}

\setlength{\textheight}{9.5in}


% ==== MACROS ==== %
\newcommand{\proof}[1]{
{\noindent {\it Proof.} {#1} \rule{2mm}{2mm} \vskip \belowdisplayskip}
}

\newtheorem{lemma}{Lemma}[section]
\newtheorem{theorem}[lemma]{Theorem}
\newtheorem{claim}[lemma]{Claim}
\newtheorem{definition}[lemma]{Definition}
\newtheorem{corollary}[lemma]{Corollary}
\lstset{moredelim=[is][\bfseries]{[*}{*]}}

% ==== DOCUMENT PROPER ==== %
\begin{document}

% --- Header Box --- %
\setlength{\fboxrule}{.5mm}\setlength{\fboxsep}{1.2mm}
\newlength{\boxlength}\setlength{\boxlength}{\textwidth}
\addtolength{\boxlength}{-4mm}
\def\ind{\hspace*{0.3in}}
\def\gap{0.2in}

\begin{center}\framebox{\parbox{\boxlength}{\bf
      Long Class Name \hfill Problem Set X\\
      ClassCode X Season 20XX \hfill Due Time DayName, Month DayNumber
}}
\end{center}

% --- Name Fields --- %
\begin{center}
  name: Charles Lai
  \ind
  netid: cjl223
  \ind
  collaborators: 
\end{center}

% --- Question --- %
\section{Problem Statement}
\section{Reasoning and Commentary}
\section{Algorithm}
\subsection{Pseudo-code}
\begin{lstlisting}
[*While*] there is existence
    [*If*] existence is pain then:
        relax
    [*Else*] chill
        [*If*] cold then:
            warm
        [*Else*] chill some more
        [*Endif*]
    [*Endif*]
[*Endwhile*]
[*Return*] relaxation
\end{lstlisting}

\section{Proof of Correctness}
\subsection*{Completion:}
\subsection*{Optimality:}
\begin{lemma} This is a claim with a list of equations:
\begin{enumerate}
\item $N_x = (4 - 0)/20 = 1/5$
\item $N_y = (10 - 0)/20 = 1/2$
\item $N_z = \sqrt{(1 - (1/5)^2 - (1/2)^2)} = \sqrt{0.71} \approx .843$
\end{enumerate}
\end{lemma}
\proof{This is a proof}

\section{Running Time Analysis}
This is a matrix!
$$R^{-1} = \begin{bmatrix}
       1       & 0           & 0         & 0\\[0.3em]
       0       & -0.8        & -0.6      & 0\\[0.3em]
       0       &  0.6        & -0.8      & 0\\[0.3em]
       0       & 0           & 0         & 1
     \end{bmatrix}$$
     
We can use the same logic in 4.1.1 to show that the gradient with respect to the weights $w$ is also a flipped convolution over a separate line equation $\tilde{x}$:
\[ \frac{\partial L}{\partial w} = \frac{\partial L}{\partial y} \star \tilde{x} \]

and cool picture:

% \end{figure}
% \begin{figure}[H]
% \centering
% \includegraphics[scale=x_size_float]{filename.ext}
% \caption{File Caption}
% \end{figure}

\end{document}
